% slide 30: tell about usecases / applications for our model
\begin{frame}
  \frametitle{Model Applications}
  \begin{itemize}
    \item Overlinking in wikipedia
    \item Recommender system for linking by editors
    \item Improve website readability and navigability
    \item A ``Continue to'' section for readers
    \item ...
  \end{itemize}
\end{frame}

\begin{frame}
  \frametitle{Future Work}
  Possible improvements for the model:
  \begin{itemize}
    \item Improve feature selection model/metrics
    \item Increase amount of article datasets
    \item Increase amount of features
    \item Implement L2R algorithms or modify existing ones
  \end{itemize}

\end{frame}


% slide 31 - .. : Inspired by the study regulation, explain what we have done in this project

\begin{frame}
  \frametitle{Conclusion}
  \begin{itemize}
    \item Constructing Ground Truth
  	\item Identifying and Extracting Features
    \item 3 Training Scenarios
      \begin{itemize}
    	\item 21 Features
        \item Non-Zero InfoGain
    	\item Top 6 Features
  	   \end{itemize}
    \item Evaluate Learning to Rank Models.
  \end{itemize}

\end{frame}

\begin{frame}
  \frametitle{Research Questions}
  \begin{enumerate}
	\item Which features of the articles influence the click frequency of links the most?
	\item To what extent can we use the Learning to Rank algorithm to predict relevance of links?
	\item Can Learning to Rank help with reducing overlinking problem?
  \end{enumerate}
\end{frame}


\begin{frame}
  \frametitle{Study Regulation}
	Objectives Reached:
       \begin{itemize}
    	\item Deep understanding of how to apply Advanced Machine Intelligence models.
        \item Construction of decision support system.
  	   \end{itemize}
    Skills Reinforced:
       \begin{itemize}
    	\item Identification of problems with large and complex data sets.
  	   \end{itemize}



  
\end{frame}


