\subsection{Ground Truth}
\label{sec:groundtruth}

James Kobielus \cite{kobielus}, from IBM Big Data and Analytics Hub, describes ground truth as ``a golden standard to which the learning algorithm needs to adapt''. In most cases, a training dataset labeled by human experts is needed to provide data patterns for the learning algorithm to use as baseline. This type of machine learning is referred to as supervised learning. The other two main approaches, unsupervised learning and reinforcement learning, aim to automate this process from the data itself, not labeled by humans.

In our case we apply a supervised learning model using a ground truth dataset constructed primarily from the clickstream data query logs. We use a subset of 1000 articles having the most outgoing clicks. We refer to these articles as being the \textit{prominent} articles. Each of these prominent articles contain links to other articles which add to a total of 144k uniquely referenced articles and 283k article $(p,\beta)$ pairs where $p$ is a prominent article and $\beta$ is a \textit{prominent-linked} article.

%feature table has to be here so that it can be displayed on the next page
\newcolumntype{M}[1]{>{\raggedright}m{#1}}
\begin{table*}[t!]
\caption{Summary of all features used for L2R algorithm}
\centering
\label{features_tab}
\begin{tabular}{lM{0.5\textwidth}l}

\centering
\textbf{Feature name} & \textbf{Description} & \textbf{Section} \\
\toprule
Popularity of resource & Processed measure on article views coming from external search engines and social media. & \ref{popularity of resource}\\
\midrule
Popularity percentile rank & Measure extracted from local distributions of article views coming from external search engines and social media. & \ref{popularity percentile rank}\\
\midrule
Link position & Position of the link deduced from its distance to the beginning of the article. & \ref{link position}\\
\midrule
Link order & Relative position in the article according to the order of appearances of all links. & \ref{link order}\\
\midrule
Link count & Number of occurrences of the link in the article. & \ref{link count}\\
\midrule
Community membership & Binary value detecting if two articles belong to the same "community" (i.e. to the same cluster of articles). & \ref{community membership}\\
\midrule
Symmetric linking & Binary value detecting if two articles both link one to each other. & \ref{symmetric linking}\\
\midrule
Hits and PageRank & Score based on the number of edges in the graph representation and on the neighbours score. Give an idea of how general the article topic is. & 
\ref{hits and pagerank}\\ 
\midrule
Relatedness & Relatedness of two articles computed on the number of links (both outgoing and incoming links) they have in common. & \ref{relatedness obtained from links}\\
\midrule
Textual similarity & Semantic comparison between two article text content implemented using cosine similarity on TF-IDF vectors. & \ref{textual similarity}\\
\midrule
Title similarity & Similarity between two article titles implemented using Jaccard coefficient on the title words. & \ref{title similarity} \\
\bottomrule
\end{tabular}
\end{table*}

The clickstream data does not contain pairs where the number of referrals from $p$ to $\beta$ is below 10 and therefore we supplement our ground truth data using the Wikipedia article dumps for the same period of time as the recording of clickstream data. In this way, we can extract article pairs that do not appear in the clickstream data because of the 10-click limit (set to protect privacy) -- this adds a substantial amount of ``unpopular'' article links to our set so that we avoid a bias towards more clicked links.