\section{Introduction}

As technology becomes part of all aspects of human nature and new trends such as the internet of things connect everyday objects to the internet, the amount of data stored in the digital universe has been growing at an outstanding rate. Consequently, processing and analysing data sets has become increasingly difficult and techniques have been developed in the fields of machine learning, big data, and others to facilitate the use of data. \\
One of the many problems that arises from this growth of digital capacity is information retrieval. We can see examples of this with search engines, recommender systems, bioinformatics, and many more, where it is necessary to rank data sets based on relevance to a certain query taking into account multiple features which influence the relevance of each option. For cases like this ones, algorithms like Learning to Rank \\
One interesting case where Learning to Rank (L2R) could be applied is to extract the importance of links in Wikipedia articles to prevent Overlinking; Containing excessive number of links, making it difficult to likely to aid the user \cite{missing_links}. Wikipedia is an internet encyclopaedia with more than 38 million articles in over 250 different languages of semi-structured information. It is also the most popular wiki-based website, and is ranked by Alexa as the \#6 most popular website in the internet. It allows the collaborative modification of its articles by the users, which is one of the main reasons Wikipedia has grown to such an impressive size. \\
Being ``the free encyclopedia that anyone can edit'' has many advantages and disadvantages, and one of the disadvantages is, as mentioned before, Overlinking. As mentioned in a study by Ashwin Paranjape et al. \cite{paranjape}:  ``in the English Wikipedia, of all the 800,000 links added ... in February 2015, the majority (66\%) were not clicked even a single time in March 2015, and among the rest, most links were clicked only very rarely''. Since most of the editing of articles is done manually, finding and removing useless links to other articles is hard to do and editors do not focus on this. Given this case, we would like answer the following questions.

\begin{itemize}
\item Is it possible reduce the amount of links in Wikipedia articles by ranking the most relevant links based on a set of the most important features?
\item How could the features for the L2R algorithm be chosen for maximum effectiveness?
\item Will reducing the number of links in a Wikipedia article improve the readability of an article and aid the reader in finding interesting and relevant links?
\end{itemize}
