\begin{abstract}
Wikipedia is one of the fastest growing websites and a primary source of information in the internet. Being a wiki, its content is crowd-sourced by the users. This has many benefits and it is one of the main reason it has grown to reach more than 5 million articles in its English version. It also raises some issues, like the overlinking of articles, which are difficult to deal with by editors. In this article, we apply Learning to Rank algorithms to evaluate the click frequency of links in an effort to distinguish the most useful links for users. To accomplish this, we develop a ground truth which serves as baseline for our algorithm and compare multiple link features to implement the most advantageous ones. The results show an 86.2 \% accuracy with the top 6 most useful features and 87.7 \% with the complete set. Considering these results, we outline a solution to the overlinking problem. By removing the most inadequate links, we suggest that readability of Wikipedia articles could be improved while preserving most of its useful links.

\end{abstract}

\textbf{Keywords:}
Learning to Rank, Supervised Learning, Wikipedia, Overlinking, Machine Learning.
