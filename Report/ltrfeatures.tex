\section{Features}
This section defines each feature used in the implementation of Learn to Rank algorithm.

\subsubsection{Link Position}
Our intuition behind the Link Position feature is based on two reasons. The first one is that given the way Wikipedia articles are structured, the most general description of the article is placed in the first few paragraphs before the table of contents. This section of the Wikipedia article is called the lead \cite{lead}. As described in the Wikipedia manual of style, for many people, the lead section is the only section they will read since it summarises the entire article. Later paragraphs only dive deeper into the topics outlined in the description. \\
The second motivation is the way people read web pages. As explained by Jakob Nielsen  \cite{nielsen}, one of the leaders in human-computer interaction, on average, users have time to read at most 28\% of the words of a website. Additionally, most attention is given to the top portion of a page and later sections are merely skimmed through. People looking for different article that is somewhat related to the one currently being read might be interested in more general concepts as they contain the searched term. As explained above, more general terms happen to be heavily abundant in the first portion of Wikipedia article. \\
To be able to measure this feature, we count the number of characters preceding the occurrence of the link. The text of the articles is taken from the wikipedia dumps previously described and links to other wikipedia articles are found using regular expressions. As expected, links to images, external sources, etc. are ignored. Due to the way this feature is extracted, there might be slight discrepancies in feature value and the exact number of characters preceding the links. \\

\subsubsection{Link Order}
Similar to the previous feature, Link Order is based on the fact that a wikipedia article's initially describe the topic in general terms and the hypothesis?? that the probability of clicking a link is higher the closer it is to the first term. While the link position captures more of a distance between links and their spread, link order is a simplified version of it. It conveys less information, but in a much more straightforward manner. \\
This feature is also extracted from the Wikipedia dumps by counting the number of links in the article. This feature captures the position of link relative to all the other links contained in the same article. The value $n$ means that the link is the $n^{\text{th}}$ link in the article. \\

\subsubsection{Community Membership}
not edited \\
This feature captures notion of two article being in the same community of articles. The communities are computed from the graph representation of Wikipedia $G(V,E)$, where $V$ is a set of articles and $E$ is set of links between them. In this scenario community is $(V?, E?) = G? \subseteq G$ such that $|E?| \ge |\{ \{u,v\} \in E\setminutE? | u \in V? \vee v \in V? \}|$. Communities are very implicit way of clustering articles on Wikipedia that captures emerging property of interconnected articles.  \\

We believe this is a promising feature in cases where readers are searching for a specific article. This is because
This feature looks promising especially in case when someone is searching for the specific article. Then related articles from the same community might be a good place to look at and will highly likely contain desired article. Also user reading an article has shown an interest in specific topic and might want to broaden and deepen his or her knowledge of it.

\subsubsection{Symmetric Linking}
This feature captures the notion of two article being interconnected in both directions. Formally, a link $(A,B)$ between article A and B is symmetric if and only if link $(B,A)$ also exists. Symmetric linking indicates, in some cases, that there exists an important relevance between said articles or highly related topics are being discussed. Examples of this article relationship includes competing presidential candidates, sports team rivals, movies and its actors, etc. As expected, it is common for users to demonstrate interest in these kind of relations between articles. By looking and the article relationships, we found in one sample of the most visited articles that 74.9\% of the links were non-symmetric and 25.1\% were symmetric.

\subsubsection{HITS and PageRanks}


\lipsum[4] % Dummy text
