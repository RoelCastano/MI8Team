%%%%%%%%%%%%%%%%%%%%%%%%%%%%%%%%%%%%%%%%%
% Journal Article
% LaTeX Template
% Version 1.3 (9/9/13)
%
% This template has been downloaded from:
% http://www.LaTeXTemplates.com
%
% Original author:
% Frits Wenneker (http://www.howtotex.com)
%
% License:
% CC BY-NC-SA 3.0 (http://creativecommons.org/licenses/by-nc-sa/3.0/)
%
%%%%%%%%%%%%%%%%%%%%%%%%%%%%%%%%%%%%%%%%%

%----------------------------------------------------------------------------------------
%	PACKAGES AND OTHER DOCUMENT CONFIGURATIONS
%----------------------------------------------------------------------------------------

\documentclass[twoside]{article}

\usepackage{graphicx}

\usepackage{lipsum} % Package to generate dummy text throughout this template

\usepackage[sc]{mathpazo} % Use the Palatino font
\usepackage[T1]{fontenc} % Use 8-bit encoding that has 256 glyphs
\linespread{1.05} % Line spacing - Palatino needs more space between lines
\usepackage{microtype} % Slightly tweak font spacing for aesthetics

\usepackage[hmarginratio=1:1,top=32mm,columnsep=20pt, left=15mm]{geometry} % Document margins
\usepackage{multicol} % Used for the two-column layout of the document
\usepackage[hang, small,labelfont=bf,up,textfont=it,up]{caption} % Custom captions under/above floats in tables or figures
\usepackage{booktabs} % Horizontal rules in tables
\usepackage{float} % Required for tables and figures in the multi-column environment - they need to be placed in specific locations with the [H] (e.g. \begin{table}[H])
\usepackage{hyperref} % For hyperlinks in the PDF
\hypersetup{
    colorlinks=true,
    citecolor = red,
    urlcolor = blue
}

\usepackage{lettrine} % The lettrine is the first enlarged letter at the beginning of the text
\usepackage{paralist} % Used for the compactitem environment which makes bullet points with less space between them



\usepackage{amsmath} % Defines some usefull math symbols
\usepackage[
backend=biber      % if we want unicode
,style=numeric
,bibencoding=utf8   % this is necessary only if bibliography file is in different encoding than main document
,maxnames=3 % maximum of three auhor names
,minnames=1 % minimum of one author (if there are more than 3 authros, only the first one is printed and "et. al." added.)
,sorting=none
,sortcites=true
,doi=true % print dois
,url=true % always print urls if present
,isbn=true % print is[bsr]n numbers
,urldate=comp % better url date format
]{biblatex}
\addbibresource{bibliography.bib}

% With 'final' over 'draft' triggers error on notes in final report
\usepackage[footnote,draft,english,silent,nomargin]{fixme}

\usepackage{abstract} % Allows abstract customization
\renewcommand{\abstractnamefont}{\normalfont\bfseries} % Set the "Abstract" text to bold
\renewcommand{\abstracttextfont}{\normalfont\small\itshape} % Set the abstract itself to small italic text

\usepackage{titlesec} % Allows customization of titles
\titleformat{\section}[block]{\large\scshape\centering}{\thesection.}{1em}{} % Change the look of the section titles
\titleformat{\subsection}[block]{\large}{\thesubsection.}{1em}{} % Change the look of the section titles

\usepackage{fancyhdr} % Headers and footers
\pagestyle{fancy} % All pages have headers and footers
\fancyhead{} % Blank out the default header
\fancyfoot{} % Blank out the default footer
\fancyfoot[RO,LE]{\thepage} % Custom footer text

%----------------------------------------------------------------------------------------
%	TITLE SECTION
%----------------------------------------------------------------------------------------

\title{\vspace{-15mm}\fontsize{24pt}{10pt}\selectfont\textbf{Super Awesome Wikipedia Hops Fairness Article}} % Article title
\author{
\large
\textsc{Blandine Seznec, Philip Thruesen, Jaroslav Cechak, and Roel Castano}\\[2mm] % Your name
\normalsize {\{bsezne16, pthrue15, jcheca16, rcasta15\}@student.aau.dk} % Your email address
}
\date{\today}
%----------------------------------------------------------------------------------------

\begin{document}

\maketitle % Insert title

\thispagestyle{fancy} % All pages have headers and footers


% Abstract
\begin{abstract}
Wikipedia is one of the fastest growing websites and a primary source of information in the internet. Being a wiki, its content is crowd-sourced by the users. This has many benefits and it is one of the main reason it has grown to reach more than 5 million articles in its English version. It also raises some issues, like the overlinking of articles, which are difficult to deal with by editors. In this article, we apply Learning to Rank algorithms to evaluate the click frequency of links in an effort to distinguish the most useful links for users. To accomplish this, we develop a ground truth which serves as baseline for our algorithm and compare multiple link features to implement the most advantageous ones. The results show an 86.2 \% accuracy with the top 6 most useful features and 87.7 \% with the complete set. Considering these results, we outline a solution to the overlinking problem. By removing the most inadequate links, we suggest that readability of Wikipedia articles could be improved while preserving most of its useful links.

\end{abstract}

\textbf{Keywords:}
Learning to Rank, Supervised Learning, Wikipedia, Overlinking, Machine Learning.


%----------------------------------------------------------------------------------------
%	ARTICLE CONTENTS
%----------------------------------------------------------------------------------------

\begin{multicols}{2} % Two-column layout throughout the main article text

%------------------------------------------------
% INTRODUCTION SECTION

\section{Introduction}

As technology becomes an embedded part of everyday life and new trends such as the internet of things connect everyday objects to the internet, the amount of data stored in the digital universe has been growing at an outstanding rate. Consequently, processing and analyzing these datasets has become increasingly difficult and consequently multiple techniques have been developed in the fields of machine learning, data mining and others, to facilitate the use of this information.
% * <philip@thruesen.dk> 2016-05-31T20:43:13.004Z:
%
% > big data
%
% Is "big data" a field? Isn't it more "data mining" we should mention
%
% ^ <roelcastanomoreno@gmail.com> 2016-06-01T08:07:31.204Z:
%
% agree too
%
% ^ <jarekcechak@gmail.com> 2016-06-02T07:32:57.431Z.
% * <philip@thruesen.dk> 2016-05-31T20:36:18.745Z:
%
% perhaps "part of ALL human nature" is a pretty harsh opinion? What about "an embedded part of everyday life"
%
% ^ <roelcastanomoreno@gmail.com> 2016-06-01T08:07:14.174Z:
%
% agreed
%
% ^ <jarekcechak@gmail.com> 2016-06-02T07:33:07.172Z.

One of the many problems that arises from this growth in digital capacity is retrieving information in a form that is useful to the users. Simply retrieving relevant information is not enough, since it can span a significant amount of data. We can see examples of this with search engines, recommender systems, and bioinformatics, where it is necessary to provide users with the relevant data based on relevance to a certain query and sort it accordingly. The Learning to Rank algorithm is designed to solve this issue by taking into account multiple features which influence the relevance of each element and extensive research has been done to improve it over the past decade.

One interesting case where Learning to Rank (L2R) could be applied to is in determining the value of certain links in Wikipedia articles to help users better navigate the encyclopedia by preventing overlinking of articles. Wikipedia is an internet encyclopedia with more than 38 million articles in over 250 different languages of semi-structured information \cite{wikistats}. It is also the most popular wiki-based website, and is ranked by Alexa as the \#6 most popular website on the internet \cite{alexa}. It allows collaborative modifications of its articles by the users, which is one of the main reasons Wikipedia has grown to such an enormous size.
% * <philip@thruesen.dk> 2016-05-31T20:56:52.080Z:
%
% > which is one of the main reasons Wikipedia has grown to such an enormous size
%
% Probably also need source/reference on this
%
% ^ <roelcastanomoreno@gmail.com> 2016-06-02T07:46:40.923Z.
% * <philip@thruesen.dk> 2016-05-31T20:53:50.897Z:
%
% > Wikipedia is an internet encyclopedia with more than 38 million articles in over 250 different languages of semi-structured information. It is also the most popular wiki-based website
%
% Needs reference
%
% ^ <roelcastanomoreno@gmail.com> 2016-06-02T07:46:56.385Z.

\begin{figure}[H]
\centering
\includegraphics[width=0.49\textwidth]{images/concept}
	\caption{The main article, ``Coffee'', links to multiple articles including the 3 illustrated. These referenced articles, or more specifically, their links appearing in the main article, we attempt to rank according to predicted click frequency.} 
 \label{fig:concept}
\end{figure}
% * <philip@thruesen.dk> 2016-05-31T21:39:19.614Z:
%
% From here and to the end of the introduction I have a bunch of changes. - hope they make sense
%
% ^.
Being ``the free encyclopedia that anyone can edit'', as Wikipedia's slogan suggests, has many advantages and disadvantages where an example of the latter is \textit{overlinking}, further explained below. From an analysis by Ashwin Paranjape et al. \cite{paranjape}:  ``in the English Wikipedia, of all the 800,000 links added to the site in February 2015, the majority (66\%) were not clicked even a single time in March 2015, and among the rest, most links were clicked only very rarely''. Since most of the editing of articles is done manually, a lot of links are added based on individual author preferences which are not always strictly according to the Wikipedia Manual of Style \cite{lead} which recommend that you insert a link if it ``would help someone understand the article you are linking from''. 
% * <philip@thruesen.dk> 2016-05-31T22:42:11.254Z:
%
% The rest of the introduction is something I just added to reference the figure. Edit as you will.
%
% ^ <roelcastanomoreno@gmail.com> 2016-06-02T07:49:21.300Z.
By assuming a link is not helping anyone when not clicked over a significant amount of time we consider a problem of predicting the most valuable (and invaluable) links for the users. We approach this as a ranking problem having multiple features as input constructed on articles and their relations. Figure \ref{fig:concept} illustrates the concept of ranking referenced articles that appear as links in the article ``Coffee''. In this fictive example we predict ``Coffee bean' to be the most clicked and ``Berry'' the least. The bottom of this ranked list are potential candidates of links to be removed due to overlinking, however, our work only focuses on ranking and making such decisions is out of this article's scope. Having this in mind, we would like answer the following questions.
\begin{itemize}
\item Which features of the articles influence the click frequency of links the most?
% * <philip@thruesen.dk> 2016-05-31T21:43:39.678Z:
%
% > click frequency
%
% I changed value to 'click frequency'. I think its the most precise description. 'value' is pretty subjective.
%
% ^ <roelcastanomoreno@gmail.com> 2016-06-02T07:49:25.842Z.
\item To what extent can we use the L2R algorithm to predict relevance of links?
\item Can L2R help with reducing overlinking problem?
% * <jarekcechak@gmail.com> 2016-05-31T21:49:37.871Z:
%
% Is the third question OK? Can we answer it in the report?
%
% ^ <roelcastanomoreno@gmail.com> 2016-06-02T07:49:27.223Z.
\end{itemize}



\section{Related Work}
Previous work has been done with link structure in Wikipedia from which we acquired key concepts for this article. \cite{learning_link} explores the issue of disambiguation and detection of possible links in external texts. In addition to the main focus in automatic cross-reference of external articles, this paper provides an understanding of certain techniques used to detect potential links in articles and the proper reference (disambiguation) of terms in wikipedia. Detecting links relies in a machine learning algorithm similar to the one applied in this article which weights in different features of the articles to provide a score to all potential links and chose the final ones. Examples of features used in this implementation include link probability, relatedness of topics, disambiguation confidence, and many others.  Learning to disambiguate links on the other hand, means identifying the correct meaning, for example ``crane'', as a large bird or a mechanical lifting machine, depending on the context (using unambiguous concepts for example) and probability of said word. \\
On the other hand, \cite{west} adds on the topic of identifying missing hyperlinks by utilising data sets of navigation paths from wikipedia-based games in which users find paths between articles. This is useful for studying shortest paths between target articles but we believe this does not address the issue of aiding users by presenting them with articles that might be interesting.\\
\cite{paranjape} comes closer to the goal of pointing out missing references by making use of server logs to weight the usefulness of links that are not yet implemented. By studying the user's paths through Wikipedia they find patterns which could be shortened with missing links.

%------------------------------------------------
% METHOD SECTION
\section{Method}

\subsection{Learning to Rank Algorithms}

In this work, the aim is to look at possibility to rank the links between articles on the Wikipedia by their real click-trough rate. Ranking is an essential part of informational retrieval, but it is not limited to it. Ranking became even more important in recent years when there is much more data than one can process in reasonable amount of time. In order to accomplish this task, we chose to use a family of algorithms called Learning to Rank. These algorithms brings machine learning approach into information retrieval.

\subsubsection{Ranking problem formulation}
Let $q$ be a query and let's denote an associated set of documents to the query $q$ as $\mathbf{x} = \{x_1, x_2, \ldots, x_m\}$. Every document $x_i$ has its label (relevance evaluation) $y_i$ in the set $\mathbf{y} = \{y_i\}^m_{i=1}$. The values of $y_i$ has to be from some totally ordered set $(S, \le)$.  Ranking can be view as a task of finding permutation $\pi$ on indices $\{1,2,\ldots, m\}$ given a query $q$ and its associated set of documents $\mathbf{x}$. Permutation $\pi$ must satisfy that $y_{\pi(j)} \le y_{\pi(i)}$ for all $1\le i < j \le m$. The sequence $x_{\pi(1)},x_{\pi(2)}, \ldots, x_{\pi(m)}$ is then ordering of retrieved documents according to their relevance in respect to the query $q$.

Learning to rank algorithms handles this task as a instance of supervised machine learning problem. Each document is represented by its feature vector. Let $q_i$ where $1 \le i \le n$ be the training queries and $\mathbf{x}^{(i)} = \{x_{j}^{(i)}\}_{j = 1}^{m^{(i)}}$ their associated documents, where $m^{(i)}$ is the number of associated documents for the query $q_i$. Then $\mathbf{y}^{(i)} = \{y_{j}^{(i)}\}_{j = 1}^{m^{(i)}}$ are labels for the associated documents to query $q_i$ (also called ground truth). Test set is $\mathbf{T} = \{(\mathbf{x}^{(i)},\mathbf{y}^{(i)} )\}_{i = 1}^{n}$ for the training queries. The algorithm then automatically learns a model for $\mathbf{T}$ in the form of a function $F(\mathbf{x}^{(i)})$ that approximates the real mapping $\hat{F}(x^{(i)}) = y^{(i)}$ on the training set. Such model can later be used to predict relevance of new query instances outside the training set, where the label is unknown.

The common idea behind creating the model in learning to rank algorithms, or machine learning in general, is optimisation of a loss function $L(F(\mathbf{x}^{(i)}), \mathbf{y}^{(i)})$ for $1 \le i \le n$. The loss function describes the quality of found ranking in terms of errors made compared to ground truth. In \cite{LTR4IR} and \cite{li}, loss function is used to categorise learning to rank algorithms into three groups; pointwise, pairwise, and listwise.

\subsubsection{Pointwise approach}
The pointwise algorithms treat each document (feature vector to be precise) as an standalone instance. The input for the resulting model is a feature vector and its output is predicted label. Essentially the problem is simplified to regression, classification, or  ordinal regression as each document is treated independently as a point in the feature space. Loss functions corresponding regression, classification, or ordinal loss functions. Limitation pointed out in \cite{LTR4IR} is assumption that relevance is absolute and does not depend on the query.

\subsubsection{Pairwise approach}
The pairwise algorithms always look at a pair of documents. The input for the model is a pair of feature vectors and the output is their relative preference, i.e. if the first from the pair should be ranked higher that the second one or vice versa. The loss function in this approach measures discrepancy between preference predicted by the model and actual order in the ground truth.

\subsubsection{Listwise approach}
The listwise algorithms look at whole document list as a whole. The input for the model is list of feature vectors and the output is either list of labels or a permutation. This approach is the only one where the loss function directly measures the final position/rank of documents.

\subsubsection{RankLib}
In this work a library called RankLib\footnote{RankLib source and binary files are available in Sourceforge repository at \url{https://sourceforge.net/p/lemur/wiki/RankLib/}} (a part of The Lemur project \cite{lemur}) has been used for performing learning to rank. This library implements the following algorithms as stated in \cite{ranklib}.
\begin{itemize}
\item MART \cite{MART} (pointwise)
\item RankNet \cite{RankNet} (pairwise)
\item RankBoost \cite{RankBoost} (pairwise)
\item AdaRank \cite{AdaRank} (listwise)
\item Coordinate Ascent \cite{CoordinateAscent} (listwise)
\item LambdaMART \cite{LambdaMART} (listwise)
\item ListNet \cite{ListNet} (listwise)
\item Random Forests \cite{RandomForests} 
\end{itemize}

\section{Data Sets}

There are multiple data sets and sources available which facilitate the interaction with Wikipedia, and due to the massive amount of data managed by wikipedia, it is important to have an organised and manageable representation of this data in our own server for processing. The most significant data source is the English Wikipedia database dump, which is released once a month, and contains the text and metadata of current revisions of all articles as XML files. Most features rely in some way or another in the database dumps since it supplies the structure and content of the articles. Wikipedia also releases the page views and page counts for all articles. \\
Another source that proved to be useful for the formation of our data sets was DBpedia, which provides a wide range of datasets for every wikipedia language on more than 10,000 articles. Given that we used a small sample of the top articles in the english version of Wikipedia, it was possible to make use of this data. Some examples of the data sets include page ids, page length, categories, links, etc. \\
Last but not least, the Wikipedia Clickstream \cite{wulczyn} project gives pairs of articles describing user navigation and includes raw counts on the volume of traffic through the specific article pairs. Typical referral sites like Google or Facebook are also included and crawler-traffic has been attempted filtered from the raw data. As explained later in the article, we build the ground truth for our model primarily using this data set.

\subsection{Ground Truth}

James Kobielus \cite{kobielus}, from IBM Big Data and Analytics Hub, describes ground truth as ``a golden standard to which the learning algorithm needs to adapt''. In most cases, a training data set labeled by human experts is needed to provide data patterns for the learning algorithm to use as baseline. This type of machine learning is referred to as supervised learning. The other two main approaches, unsupervised learning and reinforcement learning, attempt to automate the distillation of knowledge from data not previously labeled by human experts. \\

In our case to apply supervised learning, we built the ground truth for our model primarily using the Clickstream data. We use a subset of articles consisting of the 1000 articles having the most outgoing clicks. We refer to these articles as being \textit{prominent articles}. Each of these prominent articles contain links to other articles so that our ground truth dataset contains 186k uniquely referenced articles and 343k article \fxnote{Heyo} $(A,B)$ pairs where $A$ is a prominent article and $B$ is a prominent-linked article. \\

The clickstream data does not contain pairs were the navigational traffic is below 10 referrals and therefore we use the Wikipedia article dump for the same period of time where we can extract links that does not appear in the clickstream data. A large fraction of existing links are having less than 10 clicks. \\

INCOMPLETE


\subsection{Notation}
We denote our set of prominent articles $P = \{ p_1, ..., p_M \}$ where $p_i$ is one of $M$ prominent articles. Belonging to the $i^{\text{th}}$ prominent article is a set $B_i$ of linked articles: $B_i = \{ \beta_1, ..., \beta_{N_i} \}$ where $N_i$ is the number of referenced articles for the $i^{\text{th}}$ prominent article.

Click counts (popularity) for an article pair $(p, \beta)$ is denoted $click(p, \beta)$. Eq. \ref{eq:pop_prom} defines the overall popularity of an arbitrary prominent article $p_k$.
\begin{align}
click(p_k) = \sum\limits_{j=1}^{|B_k|} click(p_k, \beta_j)
\label{eq:pop_prom}
\end{align}

Eq. \ref{eq:pop_prom_link} defines the overall popularity of a prominent-linked article.
\begin{align}
click(\beta) = \sum\limits_{i=1}^{|P|} click(p_i, \beta)
\label{eq:pop_prom_link}
\end{align}

In our set of ground truth we have that:
\begin{align}
rank(p, \beta) = click(p, \beta)
\label{eq:pop_rank_relation_for_gt}
\end{align}
where a higher rank equals the level of popularity and a high rank therefore represents an important article pair.


%------------------------------------------------\\


\section{Features}
This section defines each feature used in the implementation of Learn to Rank algorithm.

\subsubsection{Link Position}
Our intuition behind the Link Position feature is based on two reasons. The first one is that given the way Wikipedia articles are structured, the most general description of the article is placed in the first few paragraphs before the table of contents. This section of the Wikipedia article is called the lead \cite{lead}. As described in the Wikipedia manual of style, for many people, the lead section is the only section they will read since it summarises the entire article. Later paragraphs only dive deeper into the topics outlined in the description. \\
The second motivation is the way people read web pages. As explained by Jakob Nielsen  \cite{nielsen}, one of the leaders in human-computer interaction, on average, users have time to read at most 28\% of the words of a website. Additionally, most attention is given to the top portion of a page and later sections are merely skimmed through. People looking for different article that is somewhat related to the one currently being read might be interested in more general concepts as they contain the searched term. As explained above, more general terms happen to be heavily abundant in the first portion of Wikipedia article. \\
To be able to measure this feature, we count the number of characters preceding the occurrence of the link. The text of the articles is taken from the wikipedia dumps previously described and links to other wikipedia articles are found using regular expressions. As expected, links to images, external sources, etc. are ignored. Due to the way this feature is extracted, there might be slight discrepancies in feature value and the exact number of characters preceding the links. \\

\subsubsection{Link Order}
Similar to the previous feature, Link Order is based on the fact that a wikipedia article's initially describe the topic in general terms and the hypothesis?? that the probability of clicking a link is higher the closer it is to the first term. While the link position captures more of a distance between links and their spread, link order is a simplified version of it. It conveys less information, but in a much more straightforward manner. \\
This feature is also extracted from the Wikipedia dumps by counting the number of links in the article. This feature captures the position of link relative to all the other links contained in the same article. The value $n$ means that the link is the $n^{\text{th}}$ link in the article. \\

\subsubsection{Community Membership}
not edited \\
This feature captures notion of two article being in the same community of articles. The communities are computed from the graph representation of Wikipedia $G(V,E)$, where $V$ is a set of articles and $E$ is set of links between them. In this scenario community is $(V?, E?) = G? \subseteq G$ such that $|E?| \ge |\{ \{u,v\} \in E\setminutE? | u \in V? \vee v \in V? \}|$. Communities are very implicit way of clustering articles on Wikipedia that captures emerging property of interconnected articles.  \\

We believe this is a promising feature in cases where readers are searching for a specific article. This is because
This feature looks promising especially in case when someone is searching for the specific article. Then related articles from the same community might be a good place to look at and will highly likely contain desired article. Also user reading an article has shown an interest in specific topic and might want to broaden and deepen his or her knowledge of it.

\subsubsection{Symmetric Linking}
This feature captures the notion of two article being interconnected in both directions. Formally, a link $(A,B)$ between article A and B is symmetric if and only if link $(B,A)$ also exists. Symmetric linking indicates, in some cases, that there exists an important relevance between said articles or highly related topics are being discussed. Examples of this article relationship includes competing presidential candidates, sports team rivals, movies and its actors, etc. As expected, it is common for users to demonstrate interest in these kind of relations between articles. By looking and the article relationships, we found in one sample of the most visited articles that 74.9\% of the links were non-symmetric and 25.1\% were symmetric.

\subsubsection{HITS and PageRanks}


\lipsum[4] % Dummy text


%------------------------------------------------

% slide 29: visualize results in bar chart. Select important algos and have 4 bars (each of eval. methods)
% slide xx: some more slides with e.g. zeros..
% slide xx: examples of ranking

\begin{frame}
  \frametitle{Results for NDCG@10}
  \begin{figure}[tbph]
    \centering
    \includegraphics[width=\linewidth]{images/results_ndcg10}
  \end{figure}
\end{frame}

\begin{frame}
  \frametitle{Results for P@191}
  \begin{figure}[tbph]
    \centering
    \includegraphics[width=\linewidth]{images/results_p191}
  \end{figure}
\end{frame}

\begin{frame}
  \frametitle{Results for PP@75}
  \begin{figure}[tbph]
    \centering
    \includegraphics[width=\linewidth]{images/results_pp75}
  \end{figure}
\end{frame}

\begin{frame}
  \frametitle{Results for PP@50}
  \begin{figure}[tbph]
    \centering
    \includegraphics[width=\linewidth]{images/results_pp50}
  \end{figure}
\end{frame}

\begin{frame}
  \frametitle{Results for PP@k for various k}
  \begin{figure}[tbph]
    \centering
    \includegraphics[width=0.95\linewidth]{images/results_pp_k}
  \end{figure}
\end{frame}

\begin{frame}
  \frametitle{Comparison of learning times}
  \begin{figure}[tbph]
    \centering
    \includegraphics[width=0.95\linewidth]{images/results_times}
  \end{figure}
\end{frame}

%------------------------------------------------

\section{Discussion}

\paragraph{Feature Selection}
Reasons for feature selection are to reduce overfitting and eliminate possible truly redundant features -- moreover, reducing the number of features can simplify the model and make it easier to interpret. In this work we focused mainly on potential performance benefits, but the results from table \ref{tab:feature_relevance} can also be useful for identifying feature subsets with potential to outperform full set. For example, taking only features with non-zero Infogain is one possible approach.
% * <philip@thruesen.dk> 2016-06-01T19:00:00.702Z:
%
% outcommented a paragraph. I dont believe using a subset of features will ever perform better than the full set unless there's overfitting which I think is very unlikely for our model.
%
% ^ <jarekcechak@gmail.com> 2016-06-02T09:10:39.850Z:
%
% This is what Nattiya told us that feature selection is usually used for
%
% ^ <jarekcechak@gmail.com> 2016-06-02T09:10:44.022Z.
% * <philip@thruesen.dk> 2016-06-01T18:56:01.192Z:
%
% > truly redundant
%
% changed conflicting to truly redundant. Change back if you disagree
%
% ^ <roelcastanomoreno@gmail.com> 2016-06-02T09:12:22.872Z.

All three methods used to evaluate the features are only heuristics and don't completely reflect true value of features. They are mainly focused on isolated importance of single feature, there might be however, complex relations between some features that will not be identified this way.

\paragraph{Applicability of Model. }
It is interesting to discuss the realistic applicability of the models generated by this research. We realize that having a fixed threshold for the number of links that may appear in an article is a na\"{\i}ve solution to the problem of overlinking. The correct number of links that may appear in an article should rather be determined by some function that considers the length of the article. Computing this is outside the scope of this work and any limits are best to be defined by Wikipedia policy makers. Depending on the use case of the suggested model, the degree to which the feature selection is needed can be reconsidered, i.e. how many resources to dedicate to the training and use of such a model.
%With that in mind and some possible improvements (Choosing a correct threshold for the number or percentage of links), 
Summarizing these thoughts, we think the model applies well to Wikipedia articles and possibly other wiki-based sites. There's always a risk that some articles `lose' a small percentage of useful links if the model was used with full automation but the benefits of improving readability through Wikipedia might out-weight the drawbacks. It may also simply be used as a tool for editors, alerting them when they are about to add a link that might confuse future readers more than it would help them.

\paragraph{Future Work}
Naturally, it is possible to expand the work done in this article. Many aspects of the research were limited by significant restraints on processing power and time availability. If more resources were available the project could be extended and up-scaled in multiple obvious ways.%, but it can be easily extended by building upon the different components.

For the future, one might choose to increase amount of articles in the dataset and change the strategy with which they are selected.  For this article, we chose the 1000 articles having the most outgoing clicks. In this way we were sure to have only articles where a ranking of referred articles was possible. Had we e.g. included articles in the prominent set without outgoing clicks we would have to rank a set of referred articles all assigned to the same label (of zero). %Another strategy to select prominent articles might be to choose one or multiple clusters of articles to generate the model or a random sample.

There's much work that could be done for the set of features. E.g. the features describing link position could be supplemented with the link position based on the visual representation after HTML had been rendered. Moreover, some features describing explicit categories could be added and the 'generality' of an article, which represents how specific or general the topic discussed is (to the general public). We have skipped construction of some potential features like the ones mentioned above due to prioritization considering our limited time.

Finally, another possibility is to evaluate other implementations of Learning to Rank algorithms or modify the existing ones. This was also outside of the scope of the project. 


%----------------------------------------------------------------------------------------
%	LIST OF ERRORS
%----------------------------------------------------------------------------------------

\newpage														% Ny side til Fixme-
\listoffixmes	


%----------------------------------------------------------------------------------------
%	REFERENCE LIST
%----------------------------------------------------------------------------------------

%\begin{thebibliography}{99} % Bibliography - this is intentionally simple in this template
	
	\bibitem[1]{learning_link}
	D. Milne and I. H. Witten,
	\newblock "Learning to Link with Wikipedia"
	\newblock \emph{ Proceeding of the 17th ACM Conference on Information and Knowledge Mining - CIKM '08 (2008): Web.}
	
	\bibitem[2]{west}
	R. West, A. Paranjape, and J. Leskovec,
	\newblock "Mining Missing Hyperlinks from Human Navigation Traces: A Case Study of Wikipedia"
	\newblock \emph{Proceedings of the 24th International Conference on World Wide Web - WWW '15 (2015): Web}
		
	\bibitem[3]{paranjape}
	A. Paranjape, R. West, L. Zia, and J. Leskovec,
	\newblock "Improving Website Hyperlink Structure Using Server Logs"
	\newblock \emph{Proceedings of the Ninth ACM International Conference on Web Search and Data Mining - WSDM '16 (2016): Web}

	\bibitem[4]{li}
	H. Li,
	\newblock "A Short Introduction to Learning to Rank"
	\newblock \emph{IEICE Transactions on Information and Systems IEICE Trans. Inf. \& Syst. E94-D.10 (2011): 1854-862}
	
	\bibitem[5]{nielsen}
	J. Nielsen,
	\newblock "How Users Read on the Web"
	\newblock \emph{Nielsen Norman Group. N.p., n.d. Web. 6 May 2008}

	\bibitem[6]{lead}
	"Lead Section"
	\newblock \emph{Wikipedia. Wikimedia Foundation.}
	\newblock https://en.wikipedia.org/wiki/ \\ Wikipedia:Manual\_of\_Style/Lead\_section
	
\end{thebibliography}

\printbibliography

%----------------------------------------------------------------------------------------

\end{multicols}

\end{document}
