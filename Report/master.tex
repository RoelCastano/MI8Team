%%%%%%%%%%%%%%%%%%%%%%%%%%%%%%%%%%%%%%%%%
% Journal Article
% LaTeX Template
% Version 1.3 (9/9/13)
%
% This template has been downloaded from:
% http://www.LaTeXTemplates.com
%
% Original author:
% Frits Wenneker (http://www.howtotex.com)
%
% License:
% CC BY-NC-SA 3.0 (http://creativecommons.org/licenses/by-nc-sa/3.0/)
%
%%%%%%%%%%%%%%%%%%%%%%%%%%%%%%%%%%%%%%%%%

%----------------------------------------------------------------------------------------
%	PACKAGES AND OTHER DOCUMENT CONFIGURATIONS
%----------------------------------------------------------------------------------------

\documentclass[twoside]{article}

\usepackage{lipsum} % Package to generate dummy text throughout this template

\usepackage[sc]{mathpazo} % Use the Palatino font
\usepackage[T1]{fontenc} % Use 8-bit encoding that has 256 glyphs
\linespread{1.05} % Line spacing - Palatino needs more space between lines
\usepackage{microtype} % Slightly tweak font spacing for aesthetics

\usepackage[hmarginratio=1:1,top=32mm,columnsep=20pt]{geometry} % Document margins
\usepackage{multicol} % Used for the two-column layout of the document
\usepackage[hang, small,labelfont=bf,up,textfont=it,up]{caption} % Custom captions under/above floats in tables or figures
\usepackage{booktabs} % Horizontal rules in tables
\usepackage{float} % Required for tables and figures in the multi-column environment - they need to be placed in specific locations with the [H] (e.g. \begin{table}[H])
\usepackage{hyperref} % For hyperlinks in the PDF

\usepackage{lettrine} % The lettrine is the first enlarged letter at the beginning of the text
\usepackage{paralist} % Used for the compactitem environment which makes bullet points with less space between them

\usepackage{abstract} % Allows abstract customization
\renewcommand{\abstractnamefont}{\normalfont\bfseries} % Set the "Abstract" text to bold
\renewcommand{\abstracttextfont}{\normalfont\small\itshape} % Set the abstract itself to small italic text

\usepackage{titlesec} % Allows customization of titles
\renewcommand\thesection{\Roman{section}} % Roman numerals for the sections
\renewcommand\thesubsection{\Roman{subsection}} % Roman numerals for subsections
\titleformat{\section}[block]{\large\scshape\centering}{\thesection.}{1em}{} % Change the look of the section titles
\titleformat{\subsection}[block]{\large}{\thesubsection.}{1em}{} % Change the look of the section titles

\usepackage{fancyhdr} % Headers and footers
\pagestyle{fancy} % All pages have headers and footers
\fancyhead{} % Blank out the default header
\fancyfoot{} % Blank out the default footer
\fancyfoot[RO,LE]{\thepage} % Custom footer text

%----------------------------------------------------------------------------------------
%	TITLE SECTION
%----------------------------------------------------------------------------------------

\title{\vspace{-15mm}\fontsize{24pt}{10pt}\selectfont\textbf{Super Awesome Wikipedia Hops Fairness Article}} % Article title
\author{
\large
\textsc{Blandine Seznec}\\[2mm] % Your name
\normalsize \href{mailto:Blandine@seznec.com}{blandine@seznec.com} % Your email address
\and
\textsc{Philip Thruesen }\\[2mm] % Your name
\normalsize \href{mailto:Philip@thruesen.com}{Philip@thruesen.com} % Your email address
\and
\textsc{Patrick Bach Andersen}\\[2mm] % Your name
\normalsize \href{mailto:Philip@thruesen.com}{Philip@thruesen.com} % Your email address
\and
\textsc{Jaroslav Cechak}\\[2mm] % Your name
\normalsize \href{mailto:Jaroslav@cechak.com}{Jaroslav@cechak.com} % Your email address
\and
\textsc{Roel Castano}\\[2mm] % Your name
\normalsize \href{mailto:Roel@castano.com}{Roel@castano.com} % Your email address
}
\date{}

%----------------------------------------------------------------------------------------

\begin{document}

\maketitle % Insert title

\thispagestyle{fancy} % All pages have headers and footers


% Abstract
\begin{abstract}
Wikipedia is one of the fastest growing websites and a primary source of information in the internet. Being a wiki, its content is crowd-sourced by the users. This has many benefits and it is one of the main reason it has grown to reach more than 5 million articles in its English version. It also raises some issues, like the overlinking of articles, which are difficult to deal with by editors. In this article, we apply Learning to Rank algorithms to evaluate the click frequency of links in an effort to distinguish the most useful links for users. To accomplish this, we develop a ground truth which serves as baseline for our algorithm and compare multiple link features to implement the most advantageous ones. The results show an 86.2 \% accuracy with the top 6 most useful features and 87.7 \% with the complete set. Considering these results, we outline a solution to the overlinking problem. By removing the most inadequate links, we suggest that readability of Wikipedia articles could be improved while preserving most of its useful links.

\end{abstract}

\textbf{Keywords:}
Learning to Rank, Supervised Learning, Wikipedia, Overlinking, Machine Learning.


%----------------------------------------------------------------------------------------
%	ARTICLE CONTENTS
%----------------------------------------------------------------------------------------

\begin{multicols}{2} % Two-column layout throughout the main article text

\section{Introduction}

As technology becomes an embedded part of everyday life and new trends such as the internet of things connect everyday objects to the internet, the amount of data stored in the digital universe has been growing at an outstanding rate. Consequently, processing and analyzing these datasets has become increasingly difficult and consequently multiple techniques have been developed in the fields of machine learning, data mining and others, to facilitate the use of this information.
% * <philip@thruesen.dk> 2016-05-31T20:43:13.004Z:
%
% > big data
%
% Is "big data" a field? Isn't it more "data mining" we should mention
%
% ^ <roelcastanomoreno@gmail.com> 2016-06-01T08:07:31.204Z:
%
% agree too
%
% ^ <jarekcechak@gmail.com> 2016-06-02T07:32:57.431Z.
% * <philip@thruesen.dk> 2016-05-31T20:36:18.745Z:
%
% perhaps "part of ALL human nature" is a pretty harsh opinion? What about "an embedded part of everyday life"
%
% ^ <roelcastanomoreno@gmail.com> 2016-06-01T08:07:14.174Z:
%
% agreed
%
% ^ <jarekcechak@gmail.com> 2016-06-02T07:33:07.172Z.

One of the many problems that arises from this growth in digital capacity is retrieving information in a form that is useful to the users. Simply retrieving relevant information is not enough, since it can span a significant amount of data. We can see examples of this with search engines, recommender systems, and bioinformatics, where it is necessary to provide users with the relevant data based on relevance to a certain query and sort it accordingly. The Learning to Rank algorithm is designed to solve this issue by taking into account multiple features which influence the relevance of each element and extensive research has been done to improve it over the past decade.

One interesting case where Learning to Rank (L2R) could be applied to is in determining the value of certain links in Wikipedia articles to help users better navigate the encyclopedia by preventing overlinking of articles. Wikipedia is an internet encyclopedia with more than 38 million articles in over 250 different languages of semi-structured information \cite{wikistats}. It is also the most popular wiki-based website, and is ranked by Alexa as the \#6 most popular website on the internet \cite{alexa}. It allows collaborative modifications of its articles by the users, which is one of the main reasons Wikipedia has grown to such an enormous size.
% * <philip@thruesen.dk> 2016-05-31T20:56:52.080Z:
%
% > which is one of the main reasons Wikipedia has grown to such an enormous size
%
% Probably also need source/reference on this
%
% ^ <roelcastanomoreno@gmail.com> 2016-06-02T07:46:40.923Z.
% * <philip@thruesen.dk> 2016-05-31T20:53:50.897Z:
%
% > Wikipedia is an internet encyclopedia with more than 38 million articles in over 250 different languages of semi-structured information. It is also the most popular wiki-based website
%
% Needs reference
%
% ^ <roelcastanomoreno@gmail.com> 2016-06-02T07:46:56.385Z.

\begin{figure}[H]
\centering
\includegraphics[width=0.49\textwidth]{images/concept}
	\caption{The main article, ``Coffee'', links to multiple articles including the 3 illustrated. These referenced articles, or more specifically, their links appearing in the main article, we attempt to rank according to predicted click frequency.} 
 \label{fig:concept}
\end{figure}
% * <philip@thruesen.dk> 2016-05-31T21:39:19.614Z:
%
% From here and to the end of the introduction I have a bunch of changes. - hope they make sense
%
% ^.
Being ``the free encyclopedia that anyone can edit'', as Wikipedia's slogan suggests, has many advantages and disadvantages where an example of the latter is \textit{overlinking}, further explained below. From an analysis by Ashwin Paranjape et al. \cite{paranjape}:  ``in the English Wikipedia, of all the 800,000 links added to the site in February 2015, the majority (66\%) were not clicked even a single time in March 2015, and among the rest, most links were clicked only very rarely''. Since most of the editing of articles is done manually, a lot of links are added based on individual author preferences which are not always strictly according to the Wikipedia Manual of Style \cite{lead} which recommend that you insert a link if it ``would help someone understand the article you are linking from''. 
% * <philip@thruesen.dk> 2016-05-31T22:42:11.254Z:
%
% The rest of the introduction is something I just added to reference the figure. Edit as you will.
%
% ^ <roelcastanomoreno@gmail.com> 2016-06-02T07:49:21.300Z.
By assuming a link is not helping anyone when not clicked over a significant amount of time we consider a problem of predicting the most valuable (and invaluable) links for the users. We approach this as a ranking problem having multiple features as input constructed on articles and their relations. Figure \ref{fig:concept} illustrates the concept of ranking referenced articles that appear as links in the article ``Coffee''. In this fictive example we predict ``Coffee bean' to be the most clicked and ``Berry'' the least. The bottom of this ranked list are potential candidates of links to be removed due to overlinking, however, our work only focuses on ranking and making such decisions is out of this article's scope. Having this in mind, we would like answer the following questions.
\begin{itemize}
\item Which features of the articles influence the click frequency of links the most?
% * <philip@thruesen.dk> 2016-05-31T21:43:39.678Z:
%
% > click frequency
%
% I changed value to 'click frequency'. I think its the most precise description. 'value' is pretty subjective.
%
% ^ <roelcastanomoreno@gmail.com> 2016-06-02T07:49:25.842Z.
\item To what extent can we use the L2R algorithm to predict relevance of links?
\item Can L2R help with reducing overlinking problem?
% * <jarekcechak@gmail.com> 2016-05-31T21:49:37.871Z:
%
% Is the third question OK? Can we answer it in the report?
%
% ^ <roelcastanomoreno@gmail.com> 2016-06-02T07:49:27.223Z.
\end{itemize}



%------------------------------------------------

\input{methods.tex}

%------------------------------------------------

% slide 29: visualize results in bar chart. Select important algos and have 4 bars (each of eval. methods)
% slide xx: some more slides with e.g. zeros..
% slide xx: examples of ranking

\begin{frame}
  \frametitle{Results for NDCG@10}
  \begin{figure}[tbph]
    \centering
    \includegraphics[width=\linewidth]{images/results_ndcg10}
  \end{figure}
\end{frame}

\begin{frame}
  \frametitle{Results for P@191}
  \begin{figure}[tbph]
    \centering
    \includegraphics[width=\linewidth]{images/results_p191}
  \end{figure}
\end{frame}

\begin{frame}
  \frametitle{Results for PP@75}
  \begin{figure}[tbph]
    \centering
    \includegraphics[width=\linewidth]{images/results_pp75}
  \end{figure}
\end{frame}

\begin{frame}
  \frametitle{Results for PP@50}
  \begin{figure}[tbph]
    \centering
    \includegraphics[width=\linewidth]{images/results_pp50}
  \end{figure}
\end{frame}

\begin{frame}
  \frametitle{Results for PP@k for various k}
  \begin{figure}[tbph]
    \centering
    \includegraphics[width=0.95\linewidth]{images/results_pp_k}
  \end{figure}
\end{frame}

\begin{frame}
  \frametitle{Comparison of learning times}
  \begin{figure}[tbph]
    \centering
    \includegraphics[width=0.95\linewidth]{images/results_times}
  \end{figure}
\end{frame}

%------------------------------------------------

\section{Discussion}

\paragraph{Feature Selection}
Reasons for feature selection are to reduce overfitting and eliminate possible truly redundant features -- moreover, reducing the number of features can simplify the model and make it easier to interpret. In this work we focused mainly on potential performance benefits, but the results from table \ref{tab:feature_relevance} can also be useful for identifying feature subsets with potential to outperform full set. For example, taking only features with non-zero Infogain is one possible approach.
% * <philip@thruesen.dk> 2016-06-01T19:00:00.702Z:
%
% outcommented a paragraph. I dont believe using a subset of features will ever perform better than the full set unless there's overfitting which I think is very unlikely for our model.
%
% ^ <jarekcechak@gmail.com> 2016-06-02T09:10:39.850Z:
%
% This is what Nattiya told us that feature selection is usually used for
%
% ^ <jarekcechak@gmail.com> 2016-06-02T09:10:44.022Z.
% * <philip@thruesen.dk> 2016-06-01T18:56:01.192Z:
%
% > truly redundant
%
% changed conflicting to truly redundant. Change back if you disagree
%
% ^ <roelcastanomoreno@gmail.com> 2016-06-02T09:12:22.872Z.

All three methods used to evaluate the features are only heuristics and don't completely reflect true value of features. They are mainly focused on isolated importance of single feature, there might be however, complex relations between some features that will not be identified this way.

\paragraph{Applicability of Model. }
It is interesting to discuss the realistic applicability of the models generated by this research. We realize that having a fixed threshold for the number of links that may appear in an article is a na\"{\i}ve solution to the problem of overlinking. The correct number of links that may appear in an article should rather be determined by some function that considers the length of the article. Computing this is outside the scope of this work and any limits are best to be defined by Wikipedia policy makers. Depending on the use case of the suggested model, the degree to which the feature selection is needed can be reconsidered, i.e. how many resources to dedicate to the training and use of such a model.
%With that in mind and some possible improvements (Choosing a correct threshold for the number or percentage of links), 
Summarizing these thoughts, we think the model applies well to Wikipedia articles and possibly other wiki-based sites. There's always a risk that some articles `lose' a small percentage of useful links if the model was used with full automation but the benefits of improving readability through Wikipedia might out-weight the drawbacks. It may also simply be used as a tool for editors, alerting them when they are about to add a link that might confuse future readers more than it would help them.

\paragraph{Future Work}
Naturally, it is possible to expand the work done in this article. Many aspects of the research were limited by significant restraints on processing power and time availability. If more resources were available the project could be extended and up-scaled in multiple obvious ways.%, but it can be easily extended by building upon the different components.

For the future, one might choose to increase amount of articles in the dataset and change the strategy with which they are selected.  For this article, we chose the 1000 articles having the most outgoing clicks. In this way we were sure to have only articles where a ranking of referred articles was possible. Had we e.g. included articles in the prominent set without outgoing clicks we would have to rank a set of referred articles all assigned to the same label (of zero). %Another strategy to select prominent articles might be to choose one or multiple clusters of articles to generate the model or a random sample.

There's much work that could be done for the set of features. E.g. the features describing link position could be supplemented with the link position based on the visual representation after HTML had been rendered. Moreover, some features describing explicit categories could be added and the 'generality' of an article, which represents how specific or general the topic discussed is (to the general public). We have skipped construction of some potential features like the ones mentioned above due to prioritization considering our limited time.

Finally, another possibility is to evaluate other implementations of Learning to Rank algorithms or modify the existing ones. This was also outside of the scope of the project. 

%----------------------------------------------------------------------------------------
%	REFERENCE LIST
%----------------------------------------------------------------------------------------

\begin{thebibliography}{99} % Bibliography - this is intentionally simple in this template
	
	\bibitem[1]{learning_link}
	D. Milne and I. H. Witten,
	\newblock "Learning to Link with Wikipedia"
	\newblock \emph{ Proceeding of the 17th ACM Conference on Information and Knowledge Mining - CIKM '08 (2008): Web.}
	
	\bibitem[2]{west}
	R. West, A. Paranjape, and J. Leskovec,
	\newblock "Mining Missing Hyperlinks from Human Navigation Traces: A Case Study of Wikipedia"
	\newblock \emph{Proceedings of the 24th International Conference on World Wide Web - WWW '15 (2015): Web}
		
	\bibitem[3]{paranjape}
	A. Paranjape, R. West, L. Zia, and J. Leskovec,
	\newblock "Improving Website Hyperlink Structure Using Server Logs"
	\newblock \emph{Proceedings of the Ninth ACM International Conference on Web Search and Data Mining - WSDM '16 (2016): Web}

	\bibitem[4]{li}
	H. Li,
	\newblock "A Short Introduction to Learning to Rank"
	\newblock \emph{IEICE Transactions on Information and Systems IEICE Trans. Inf. \& Syst. E94-D.10 (2011): 1854-862}
	
	\bibitem[5]{nielsen}
	J. Nielsen,
	\newblock "How Users Read on the Web"
	\newblock \emph{Nielsen Norman Group. N.p., n.d. Web. 6 May 2008}

	\bibitem[6]{lead}
	"Lead Section"
	\newblock \emph{Wikipedia. Wikimedia Foundation.}
	\newblock https://en.wikipedia.org/wiki/ \\ Wikipedia:Manual\_of\_Style/Lead\_section
	
\end{thebibliography}


%----------------------------------------------------------------------------------------

\end{multicols}

\end{document}
