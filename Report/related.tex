\subsection{Related Work}
\subsubsection*{Learning to Rank Approach}
Previous work has been done in determining value of content in social networks by utilizing L2R algorithms. Similar work was done by Duan et. al. \cite{l2rtwitter}, who worked in analyzing micro-blogging systems, focusing on the Twitter social network, and extracting features which could impact the relevance of a tweet. 

\subsubsection*{Wikipedia}
Another already explored area is the link structure in Wikipedia, from which we acquired key concepts for this article. \cite{learning_link} explores the issue of disambiguation and detection of possible links in external texts. In addition to the main focus in automatic cross-reference of external articles, this paper provides an understanding of certain techniques used to detect potential links in articles and the proper reference (disambiguation) of terms in Wikipedia. Detecting links relies in a machine learning algorithm similar to the one applied in this article which weights in different features of the articles to provide a score to all potential links and choose the final ones. Examples of features used in this implementation include link probability, relatedness of topics, disambiguation confidence, and many others.  Learning to disambiguate links on the other hand, means identifying the correct meaning, for example ``crane'', as a large bird or a mechanical lifting machine, depending on the context (using unambiguous concepts for example) and probability of said word.

On the other hand, West et. al. \cite{west} add on the topic of identifying missing hyperlinks by utilizing datasets of navigation paths from Wikipedia-based games in which users try to find the shortest paths between articles, which is useful for understanding user navigation. They then\cite{paranjape}  point out missing references by making use of server logs to weight the usefulness of links that don't exist yet. By studying the user's paths through Wikipedia, they find patterns which could be shortened with missing links.